\section{Yleistä}
% analyysi, kompleksiluvut
% todo: yhtenäistä taulukot
Tarve matematiikan ilmaisemiselle digitaalisessa muodossa kasvaa jatkuvasti, eikä se rajoitu pelkästään näkövammaisiin.
Monista kehittyneemmistä tekstinkäsittelyohjelmista löytyy nykypäivänä työkalut kaavojen muotoiluun ja erikoissymbolien tuottamiseen.
Tällaisia ominaisuuksia ei kuitenkaan useimmiten koeta kovin käytettäviksi, sillä kaavan rakentaminen voi vaatia paljon hiiren käyttöä ja erinäisissä valikoissa 
kiertelyä.
Jokaisen ohjelman toteuttama kaavan muokkaus on lisäksi erilainen ja joka paikassa näitä mahdollisuuksia ei oleollenkaan saatavilla.
Esimerkiksi erinäiset pikaviestimet valtaavat alaa viestinnässä, mutta niistä harvoin löytyy erillisiä keinoja matematiikan ilmaisemiseen.

%Matematiikkaa kirjoitetaan yhä enemmän tietokoneen näppäimistöllä. 
%Tämä ei päde pelkästään näkövammaisiin, vaan kaikkiin muihinkin.
%Kaavojen muotoilu ja erikoissymbolien käyttäminen löytyvät ominaisuuksina joistan tekstinkäsittelyohejlmista, mutta 
%ne eivät ole kovin käteviä ja samaa mahdollisuutta ei joka tilanteessa ole.

Tämän oppaan on tarkoitus tarjota kyseiseen ongelmaan ratkaisuna ikään kuin matematiikan translitterointi, joka on tuttu menetelmä eri kielien symbolien 
eriäväisyyksien hallinnasta.
Yhteisymmärryksen saavuttamiseksi tarvitsemme yhtenäisen ja silti tiiviin tavan esittää matematiikasta tuutuja merkintöjä kuten esimerkiksi kreikkalaisia aakkosia.
Tavoitteena on esittää asiat selkeästi ilman väärinkäsityksen mahdollisuuksia, mutta samalla mahdollisimman kompaktissa muodossa.
Toisaalta matematiikka on kehittyvä kieli, jonka käytölle ei pitäisi myöskään asettaa liian tiukkoja rajotteita.
Esimerkiksi merkintöjen lyhentämismahdollisuudet ovat tärkeässä asemassa, jotta voidaan estää liian pitkiksi tai monimutkaisisksi paisuvat kokonaisuudet.
Tällaiseen tarvitaan ennemmin joukko yleisiä käytäntöjä kuin tarkka ennalta kirjoitettu lista kaikista mdhollisista lyhenteistä.

Merkintöjä suunniteltaessa vaikutteita on otettu saman ongelman aiemmista ratkaisuyrityksistä: ohjelmointikielistä, LaTeX:sta sekä ASCIImath:sta.
Yksinkertaisen aritmetiikan ilmaiseminen on tärkeä osa mitä tahansa ohjelmointikieltä ja tältä osin merkinnät ovat usein hyvin vakiintuneita eri kielten välillä.
LaTeX on tekstin ladontaan suunnattu kieli, joka pyrkii lopputuloksen visuaaliseen täsmällisyyteen itse lähdekoodin luettavuuden ja ymmärrettävyyden 
kustannuksella.
LaTeX-koodin tehtävänä on esittää matematiikkaa painetussa muodossa. Sellaisenaan se siis sisältää itse matematiikan kannalta täysin turhaa tietoa.
Toisaalta LaTeX on niin laajalti käytetty, että yhteneväisyydet sen kanssa auttavat merkintöjen ymmärrettävyyttä.
ASCIImath puolestaan koittaa olla syntaksiltaan yksinertaisempi.
Siinä on pyritty sanallistamaan ja lyhentämään käytetyt symbolit, joten sekään 
ei oikeastaan kuvaa matematiikkaa, vaan pyrkii jäljittelemään sen visuaalista esitystä.
ASCIImathin merkinnät tuppaavat muodostumaan tarpeettoman pitkiksi tai ovat epäintuitiivisia.

Jokainen merkki esiintyy aina kontekstissa. Tästä on hyötyä siinä, ettei merkintöjen täydy olla universaalisti yksilöllisiä. Matematiikassa on jopa tyypillistä 
määritellä uusia operaattoreita tai ominaisuuksia vain tiettyä tarkoitusta varten. On siis hyvä muistaa, että kaikkea voi lyhentää tarvittaessa ja omia merkintöjä 
voi hyvin käyttää. Asia pitää vain selittää huolellisesti auki.

\section{Muuttujat ja muut symbolit}
Yksinertaiset muuttujat ilmaistaan tavallisina kirjaimina kuten esimerkiksi \verb$x$ ja \verb$y$.
Nämä ovat tyypillisesti yhden kirjaimen mittaisia ja siksi erottuvat helposti toisistaan yhteenkin kirjoitettuina.
Yleensä muuttujia erottaa kuitenkin jokin operaattori, sulku tai väli.

Kaikki erikoisemmat symbolit niiden käyttötarkoituksesta huolimatta voidaan ilmaista sanallisessa muodossa.
Esimerkiksi kreikkalaisten aakkoston ensimmäiset aakkoset voi kirjoittaa muodossa alpha, beta ja gamma.
Käytettyjen sanallisten muotojen tulisi olla mahdollisimman selkeitä ja lyhyitä.
Englannin kieliset termit ymmärretään laajemmin ja ne ovat usein myös suomen kielisiä vastineita lyhyempiä.
Yleensä tälaiset erikoismerkityksen omaavat sanat eivät ole vaarassa sekoittua muuttujina käytettyihin kirjaimiin.

Sanallisia muotoja voi myös lyhentää.
Jotkin lyhenteen ovat vakiintuneita ja siksi ymmärrettäviä sellaisenaan,
mutta uuden erikoissymbolin tai lyhenteen voi ottaa koska vain käyttöön avaamalla sen merkityksen lukijalle kirjan, luvun, tehtävänannon taikka tehtävän alussa.
Voidaan esimerkiksi sanoa, että merkitään jotain $+$-operaattoria muistuttavaa uutta operaattoria symbolilla \verb$\+$.
Lyhenteet voivat olla jopa kirjaimen mittaisia, kuten joukko-opin leikkaus \verb$\i$ ja yhdiste \verb$\u$.
Lyhyiden lyhenteiden eteen merkitään etuliitteeksi \verb$\$-merkki, jotta ne erottuisivat tavallisista muuttujista.

Ylä- ja alaindeksejä voi olla millä tahansa symbolilla.
Alaindeksi merkitään alaviivalla ja yläindeksi potenssimerkillä.
Jos symbolilla on sekä ala- että yläindeksi, suositaan järjestystä, jossa alaindeksi kirjoitetaan ensin.
Yhtä merkkiä pidemmät ala- ja yläindeksit ilmaistaan aaltosulkeiden sisällä.

Joskus muuttujia erotetaan toisistaan koristelemalla ne erilaisilla lisämerkeillä kuten hattu-merkillä.
Sillä ei sinänsä ole väliä, onko käytetty merkki juurikin hattu, joten
tällaiset tilanteet voi hoitaa lisäämällä jonkin vähemmän käytetyn merkin symbolin perään.
Esimerkiksi muuttujaa $x$ voisi koristella vaikka muotoon: \verb$x'$, \verb$x`$, \verb$x"$ tai \verb!x$!.
Sen sijaan tulisi välttää merkintöjä kuten \verb$x^$ tai \verb$x~$, jotka voivat helposti sekoittua muiden merkkien kanssa.

\section{Realiluvut}
Luvut merkitään yhteen kirjoitetuin numeroin.
Pidemmät luvut voi ryhmitellä listeillä kolmen numeron sarjoihin.
Desimaaliosa erotellaan tavalliseen tapaan pilkulla.

Negatiivinen luku merkitään viivalla, esimerkiksi $-1$ ja vastaavasti positiivisen luvun positiivisuutta voi korostaa lisäämällä plussan, $+1$.
Etumerkit ja luvut kirjoitetaan yhteen.

{\bf Murtoluvut} merkitään jakolaskuina, koska näillä ei ole mitään eroa.
Vastaavasti {\bf sekaluvun} voi ilmaista plussalla. Esimerkiksi kolme kokonaista ja yksi neljäsosa on \verb$3+1/4$.
Jos haluaa korostaa, että kyseessä on murto- tai sekaluku, sen voi kirjoittaa yhteen ja ympäröidä välein jossain kaavassa.

{\bf Prosentit} eli sadasosat merkitään loogisesti prosenttimerkillä \verb$%$ ja {\bf promillea} eli tuhannesosia merkinnällä \verb$:%$.

{\bf Ääretöntä} ilmaistaan sanalla infty ja se voidaan lyhentää merkinnällä \verb$\8$.
Ääretön-symbolin eteen voi tavallisesti merkitä etumerkin $-$ tai $+$.

Realiakselin väleejä ilmaistaan hakasulkein, joiden suunta määrää välin avoimmuuden.
{\bf Suljettu väli} pisteiden $a$ ja $b$ välillä ilmaistaan merkinnällä \verb$[a,b]$.
Vastaava {\bf avoin väli} on puolestaan \verb$]a,b[$.
Kyseessä on {\bf puoliavoin väli}, jos molemmat sulut ovat saman suuntaiset: 
oikealta avoin väli on \verb$[a,b[$ ja vasemmalta avoin \verb$]a,b]$.
Välit ovat joukkoja, joiden alkioilla on järjestys.

%Kaikki symbolit on mahdollista välttää käyttämällä sanallista ilmaisua kuten kreikkalaisten aakkosten kohdalla.
%Yleensä tälaiset erikoismerkityksen omaavat sanat eivät ole vaarassa sekoittua muuttujina käytettyihin kirjaimiin.
%Merkintöjä voi selkiyttää lisäämällä erikoismerkin eteen \verb$\$-merkin.
%Erityisesti symboleista käytettävät lyhenteet on syytä erottaa tavallisista kirjaimista.
%Lyhenteet voivat olla jopa kirjaimen mittaisia, kuten joukko-opin leikkaus \verb$\i$ ja yhdiste \verb$\u$.
%Uuden erikoissymbolin tai lyhenteen voi ottaa koska vain käyttöön avaamalla sen merkityksen lukijalle kirjan, luvun, tehtävänannon taikka tehtävän alussa.
%Voidaan esimerkiksi sanoa, että merkitään jotain $+$-operaattoria muistuttavaa uutta operaattoria symbolilla \verb$\+$.

%Kaikkia erikoissymboleja merkitään \verb$\$-alkuliitteellä aivan kuten LaTeXissa.
%Tällaisia ovat kreikkalaisten aakkosten lisäksi nimetyt funktiot (neliöjuuri), integraalit, summafunktiot ja erikoisemmat operaattorit.
%Tämä tuo vapautta merkintöjen selkiyttämiseen menettämättä kuitenkaan tuloksen täsmällisyyttä.
%Merkintöjen lyhyt avaaminen auttaa myös näihin merkintöihin tottumatonta ymmärtämään mistä on kyse.

% Yksiköt?
% 3ms
% 5m/s
% 80km/h
% 9.81m/s^2

\section{Kreikkalaiset aakkoset}
Kreikkalaiset aakkoset voi ilmaista sanallisesti esimerkiksi \verb$alpha$ ja \verb$beta$.
Näitä käytetään todella tiuhaan, joten niiden lyhentämiseen on varattu oma etuliite $~$-merkki.
Taulukossa~\ref{tbl:greek} on jokaisen kreikkalaisen aakkosen lyhenne ja LaTeX:in kanssa yhtenevä pidempi sanallinen muoto.

\begin{table}[h!]
\begin{tabular}{ l | l | l }
% alpha beta gamma delta epsilon zeta eta theta iota kappa lambda mu nu xi omicron pi rho sigma tau ypsilon phi chi psi omega
% a b c d e z g h i k l m n q o p r s t u f x y w
% unused letters: j v
Sanallinen & Lyhenne & Aakkonen \\ \hline
alpha & $\alpha$ & \verb$~a$ \\
beta & $\beta$ & \verb$~b$ \\
gamma & $\gamma$ & \verb$~c$ \\
delta & $\delta$ & \verb$~d$ \\
epsilon & $\epsilon$ & \verb$~e$ \\
zeta & $\zeta$ & \verb$~z$ \\
eta & $\eta$ & \verb$~g$ \\
theta & $\theta$ & \verb$~h$ \\
iota & $\iota$ & \verb$~i$ \\
kappa & $\kappa$ & \verb$~k$ \\
lambda & $\lambda$ & \verb$~l$ \\
mu & $\mu$ & \verb$~m$ \\
nu & $\nu$ & \verb$~n$ \\
xi & $\xi$ & \verb$~q$ \\
omicron & $o$ & \verb$~o$ \\
pi & $\pi$ & \verb$~p$ \\
rho & $\rho$ & \verb$~r$ \\
sigma & $\sigma$ & \verb$~s$ \\
tau & $\tau$ & \verb$~t$ \\
upsilon & $\upsilon$ & \verb$~u$ \\
phi & $\phi$ & \verb$~f$ \\
chi & $\chi$ & \verb$~x$ \\
psi & $\psi$ & \verb$~y$ \\
omega & $\omega$ & \verb$~w$ \\
\end{tabular}
\caption{Kreikkalaiset aakkoset}
\label{tbl:greek}
\end{table}

\section{Kaavat ja aritmetiikkaa}
Numeroista, symboleista ja operaatioista muodostetaan kaavoja.
Kaavan eri osia erotellaan toisistaan välein tarvittaessa tai selkeyden lisäämiseksi.
Välilyöntejä käytetään pistemerkeistä eroavasti.
Välilyöntien tarkoituksena on selkiyttää ja ryhmitellä kaavoja.
Turhia välilyöntejä on syytä välttää, sillä se tekee kokonaisuuksista pidempiä.
Tarkkoja sääntöjä tästä ei anneta, sillä selkeyden määrittely eri tilanteissa olisi hankalaa.

On kuitenkin syytä huomioida seuraavat asiat:
\begin{itemize}
	\item ei välejä sulkujen sisäpuolelle
	\item väli operaattorin molemmille puolille tai sitten ei kummallekaan
	\item ei välejä jokaisen merkin ympärille, koska muuten sen merkitys katoaa
\end{itemize}

Taulukko~\ref{tbl:arithmop} luettelee yleisimpiä aritmeettisia operaattoreita.
Operaatioilla on niille sovitut laskujärjestyssäännöt.
Laskujärjestyksen muuttaminen ja lausekkeen ryhmittely tapahtuu kaarisulkeiden \verb$($ ja \verb$)$ avulla.

On syytä huomata, että sulkeita tarvitaan joskus tavallisia matematiikan merkintöjä enemmän.
Näin käy mm. potenssin ja jakolaskun kohdalla, jolloin ryhmitely tulee selväksi asettelusta. 
Seuraavat esimerkit selventävät eroja:
\begin{itemize}
	\item useampi terminen nimittäjä on aina sulkeissa: $\frac{1}{x y}$ merkitään \verb$1/(x*y)$.
	\item Pelkkää tuloa sisältävä osoittaja ei tarvitse sulkeita: $\frac{x y}{z}$ voidaan kirjoittaa \verb$x*y/z$.
	\item Summan tai erotuksen sisältämät osoittajat ja nimittäjät pitää aina laittaa sulkeisiin: $\frac{x+y}{z}$ on välttämättö merkittävä muotoon 
	\verb$(x+y)/z$.
\end{itemize}

\begin{table}[H]
\begin{tabular}{ l|l|l }
\multicolumn{2}{c}{Operaatio} & Merkintä \\ \hline
Summa & $x+y$ & \verb$x+y$ \\
Erotus & $x-y$ & \verb$x-y$ \\
Tulo & $x y$ & \verb$x*y$, \verb$xy$ \\
Jako & $\frac{x}{y}$ & \verb$x/y$ \\
Potenssi & $x^y$ & \verb$x^y$ \\
Itseisarvo & $|x|$ & \verb$|x|$ \\
Yhtäsuuruus & $x=y$ & \verb$x=y$ \\
Erisuuruus & $x\neq y$ & \verb$x!=y$ \\
Likimäärin, noin & $x\approx y$ & \verb$x~=y$ \\
Pienempi kuin & $x < y$ & \verb$x<y$ \\
Pienempi tai yhtäsuuri & $x \leq y$ & \verb$x<=y$ \\
Suurempi kuin & $x > y$ & \verb$x>y$ \\
Suurempi tai yhtäsuuri & $x \geq y$ & \verb$x>=y$ \\
paljon pienempi kuin & $x << y$ & \verb$x << y$ \\
paljon suurempi kuin & $x >> y$ & \verb$x >> y$ \\
\end{tabular}
\caption{Yleisimpiä aritmeettisia operaattoreita}
\label{tbl:arithmop}
\end{table}

Taulukko~\ref{tbl:arithmfunc} esittelee yleisimpiä aritmeettisia funktioita, joihin voi viitata tunnetulla nimellä
tai vahtoehtoisesti suoraan käyttäen operaation alkuperäistä muotoa.
Esimerkiksi juuria voi täysin luontevasti ilmaista potensseina. 
Neliöjuuri (sqrt; square root) on \verb$x^(1/2)$, kuutiojuuri (cbrt; cube root) \verb$x^(1/3)$ ja $n$:nnes juuri (nrt; nth root) vastaavasti \verb$x^(1/n)$.

\begin{table}[h!]
\begin{tabular}{ l|l|l }
\multicolumn{2}{c}{Operaatio} & Merkintä \\ \hline
neliöjuuri & $\sqrt{x}$ & \verb$sqrt(x)$, \verb$\rt2(x)$ \\
kuutiojuuri & $\sqrt[3]{x}$ & \verb$cbrt(x)$, \verb$\rt3(x)$ \\
n:s juuri & $\sqrt[n]{x}$ & \verb$nrt(x)$, \verb$(x)^(1/n)$ \\
eksponenttifunktio & $e^x$ & \verb$e^x$ \\
logaritmi & $\log x$ & \verb$log x$ \\
a-kantainen logaritmi & $\log_a x$ & \verb$log_a x$ \\
\end{tabular}
\caption{Aritmeettisia funktioita}
\label{tbl:arithmfunc}
\end{table}

\begin{table}[ht]
\begin{tabular}{ l|l|l }
\multicolumn{2}{c}{Operaatio} & Merkintä \\ \hline
x on y:n tekijä & $x\mid y$ & \verb$x | y$ \\
x ei ole y:n tekijä & $x\nmid y$ & \verb$x !| y$ \\

ääretön & $\infty$ & infty, \verb$\8$ \\
plus ääretön & $+\infty$ & +infty, \verb$+\8$ \\
miinus ääretön & $-\infty$ & -infty, \verb$-\8$ \\

prosentti & $\percent$ & \verb$%$ \\
promille & $\permil$ & \verb$:%$ \\
kertoma & $n!$ & \verb$n!$ \\
\end{tabular}
\caption{mihin nämä kuuluvat}
\label{tbl:hmm}
\end{table}

\FloatBarrier

\section{Logiikka}
Taulukko~\ref{tbl:logicop} listaa loogisia operaattoreita.

\begin{table}[ht]
\begin{tabular}{ l|l|l }
\multicolumn{2}{c}{Operaatio} & Merkintä \\ \hline
ja & $a \wedge b$ & \verb$a & b$ \\
tai & $a \vee b$ & \verb$a | b$ \\
negaatio & $\neg a$ & \verb$!a$ \\
implikaatio & $a \Rightarrow b$ & \verb$a --> b$, \verb$b <-- a$ \\
ekvivalenssi & $a \Leftrightarrow b$ & \verb$a <--> b$ \\
\end{tabular}
\caption{Loogisia operaatioita.}
\label{tbl:logicop}
\end{table}

\section{Geometria}
\begin{table}[ht]
\begin{tabular}{ l|l|l }
\multicolumn{2}{c}{Kuvaus} & Merkintä \\ \hline
yhdensuuntaiset & $l \parallel m$ & \verb$l || m$ \\
ei yhdensuuntaiset & $l \nparallel m$ & \verb$l |\| m$ \\
kohtisuorat & $l \perp m$ & \verb$l -|- m$ \\
kuviot yhdenmuotoiset & $a\sim b$ & $a ~ b$ \\
kuviot yhtenevät & $a\cong b$ & $a ~= b$ \\
kulma & $\sphericalangle ABC$ & \verb$kulma ABC$ \\
\end{tabular}
\caption{Geometrian symboleja}
\label{tbl:geometry}
\end{table}

\begin{table}[ht]
\begin{tabular}{ ll|l }
\multicolumn{2}{c}{Kuvaus} & Merkintä \\ \hline
suuntajana, vektori & $\overrightarrow{AB}$ & \verb$<AB>$ \\
vektori & $\vec{a}$ & \verb$<a>$ \\
vektorin pituus, normi & $\| x\|$ & \verb$||x||$ \\
pistetulo & $\vec{a} \cdot \vec{b}$ & \verb$a . b$ \\
ristitulo & $\vec{a} \times \vec{b}$ & \verb$a \x b$ \\
sisätulo & $\langle x,y\rangle$ & \verb$<x,y>$ \\
saman suuntaiset & $\vec{a} \uparrow\uparrow \vec{b}$ & \verb$a >|> b$ \\
vastakkaissuuntaiset & $\vec{a} \uparrow\downarrow \vec{b}$ & \verb$a >|< b$ \\
$x$:n suuntainen yksikkövektori & $\vec{a}^{\circ}$ & \verb$x/||x||$ \\
\end{tabular}
\caption{Vektorimerkintöjä}
\label{tbl:vectors}
\end{table}

\section{Joukko-oppi}
% verrannollinen, noudattaa jakaumaa
% todo: täydennä joukkojen määrittelyt taulukossa
Joukkoa kuvataan alltosulkein, joiden välissä ilmaistaan joukon sisältämät alkiot.
Alkiot voi määritellä luettelemalla sen alkiot pilkuilla eroteltuina.
Vaihtoehtoinen tapa on parametrisoitu määritelmä, jossa aaltosulkeissa ensin kerrotaan alkiot määräävä muuttuja ja pystyviivan jälkeen rajaava ehto.
Takulukossa~\ref{tbl:setop} luetellaan joukon eri määrittelytapoja sekä joukkojen välisiä operaatioita.
Taulukko~\ref{tbl:numeric} listaa yleisille numeerosille joukoille annettuja symboleja.

Jotkin operaatioista, kuten leikkaus ja yhdiste, voidaan kohdistaa kahden joukon sijasta myös useaan joukkoon.
Esimerkiksi jokaisen kokonaisluvun muodostamien yksiöiden yhdiste on koko kokonaislukujen joukko: \verb$\u{x\in \Z} {x} = \Z$.
Tällaisia indeksijoukollisia operaatioita tarkastellaan lähemmin luvussa~\ref{sec:indexsets}.

\begin{table}[ht]
\begin{tabular}{ l|l|l }
\multicolumn{2}{c}{Numeerinen joukko} & Symboli \\ \hline
Luonnolliset luvut & $\mathbb{N}$ & \verb$\N$ \\
Kokonaisluvut & $\mathbb{Z}$ & \verb$\Z$ \\
Realiluvut & $\mathbb{R}$ & \verb$\R$ \\
Rationaaliluvut & $\mathbb{Q}$ & \verb$\Q$ \\
Irrationaaliluvut & $\mathbb{R} \setminus \mathbb{Q}$ & \verb$\R \m \Q$ \\
Kompleksiluvut & $\mathbb{C}$ & \verb$\C$ \\
Algebralliset luvut & $\mathbb{A}$ & \verb$\A$ \\
\end{tabular}
\caption{Numeerisia joukkoja}
\label{tbl:numeric}
\end{table}

\begin{table}[ht]
\begin{tabular}{ l|l|l }
\multicolumn{2}{c}{Seloste} & Symboli \\ \hline
tyhjä joukko & \verb${}$, \verb$\0$ & $\emptyset$ \\
joukko, yksiö & \verb${x}$ & $\{x\}$ \\
kaksio & \verb${x,y}$ & $\{x,y\}$ \\
joukko & \verb${x_1,x_2,...}$ & $\{x_1,x_2,\ldots\}$ \\
& \verb${2x | x \in \N}$ & $\{2x \mid x \in \mathbb{N}\}$ \\
Kuuluu joukkoon & \verb$x \in A$ & $x\in A$ \\
ei kuuluu joukkoon & \verb$x !\in A$ & $x\notin A$ \\
osajoukko & \verb$A \sub B$ & $A \subset B$ \\
& \verb$A \sub= B$ & $A \subseteq B$ \\
& \verb$B \sup A$ & $B \supset A$ \\
aito osajoukko & \verb$A \sub!= B$ & $A \subsetneq B$ \\
ei osajoukko & \verb$A !\sub B$ & \\
komplementti & \verb$A^c$ & $A^c$ \\
avoin osajoukko & \verb$A \subo B$ & todo \\
suljettu osajoukko & \verb$A subc B$ & todo \\
suljettu väli & $[a,b]$ & \verb$[a,b]$ \\
avoin väli & $]a,b[$ & \verb$]a,b[$ \\
puoliavoin väli & $[a,b[$ & \verb$[a,b[$ \\
& $]a,b]$ & \verb$]a,b]$ \\
pienin yläraja & $\sup A$ & \verb$sup A$ \\
suurin alaraja & $\inf A$ & \verb$inf A$ \\
pienin luku & $\min A$ & \verb$min A$ \\
suurin luku & $\max A$ & \verb$max A$ \\
alkioiden lukumäärä & $n(A)$ & \verb$n(A)$ \\
Ristitulo & \verb$A \x B$ & $A\times B$ \\
yhdiste & \verb$A \u B$ & $A\cup B$ \\
leikkaus & \verb$A \i B$ & $A\cap B$ \\
\end{tabular}
\caption{Binäärisiä joukko-opin operaatioita}
\label{tbl:setop}
\end{table}

\begin{table}[ht]
\begin{tabular}{ ll|l|l }
\multicolumn{2}{c}{Kvanttori} & Sanallisesti & Lyhenne \\ \hline
olemassaolo & $\exists x$ & exists & \verb$\e x$ \\
yksikäsitteisyys & $\exists! x$ & exists! & \verb$\e! x$ \\
todo & $\nexists x$ & !exists & \verb$!\e x$ \\
universaali, kaikilla & $\forall x$ & forall & \verb$\a x$ \\
\end{tabular}
\caption{Kvanttoreita}
\label{tbl:setq}
\end{table}

% todo: selitä kvanttoreita tekstissä myös
% {2*x | x in \N}
% {x | P(x)}
% {x \in A | P(x)}
% \U{i in \N} A_i Joukkojen A_1, A_2, ..., A_n unioni
% \I{i in \N} A_i Joukkojen A_1, A_2, ..., A_n leikkaus
\FloatBarrier

\section{Kuvaukset ja funktiot}
\begin{table}[ht]
\begin{tabular}{ ll|l }
\multicolumn{2}{c}{Kuvaus tai funktio} & Merkintä \\ \hline
kuvaus & $f: A \rightarrow B$ & \verb$f: A -> B$ \\
kuvautuu & $f \mapsto f(x)$ & \verb$x |-> f(x)$ \\
käänteiskuvaus & $f^{-1}$ & \verb$f^(-1)$ \\
Funktioiden yhdiste & $f\circ g$ & \verb$f@g$ \\
Funktion $f$ kuva & $fA$ & \verb$fA$ \\
Alkion kuva & $f{x}$ & \verb$f{x}$ \\
\end{tabular}
\caption{Kuvauksia ja funktioita}
\label{tbl:func}
\end{table}

\section{Todennäkösyyslaskenta}
Taulukko~\ref{tbl:prob} esittelee yleisimpiä todennäköisyyslaksennassa käytettäviä merkintöjä.

\begin{table}[ht]
\begin{tabular}{ ll|l }
\multicolumn{2}{l}{Symboli} & Merkintä \\ \hline
satunnaismuuttuja & $X$, $\underline{x}$ & $X$ \\
A:n todennäköisyys & $P(A)$ & $P(A)$ \\
todennäköisyys ehdollinen B:llä & $P(A \mid B)$ & $P(A | B)$ \\
komplementtitapahtuma & $\bar{A}$ & \verb$A^c$ \\
x:n keskiarvo & $\bar{x}$ & \verb$mean(x)$ \\
X:n odotusarvo & $EX$, $E(X)$, $\mu_X$ & \verb$EX$, \verb$E(X)$, \verb$~m_X$ \\
X:n keskihajonta & $DX$, $D(X)$, $\sigma_X$ & \verb$DX$, \verb$D(X)$, $~s_X$ \\
X:n varianssi & $D^2(X)$, $\sigma^2$ & \verb$D^2(X)$, \verb$\sigma^2$ \\
normaalijakauma & $N(\mu,\sigma)$ & \verb$N(~m,~s)$ \\
verrannollinen & $x\propto y$ & \verb$x ~~ y$ \\
noudattaa jakaumaa & $X\sim Y$ & \verb$X ~ Y$ \\
\end{tabular}
\caption{Todennäköisyyslaskennasta ja jakaumista}
\label{tbl:prob}
\end{table}

\FloatBarrier

\section{Vektorit ja matriisit}
Vektoreilla ja matriiseilla kuvataan useampiulotteisia muuttujia. 
Oikeastaan vektorit ovat yksirivisiä tai -sarakkeisia matriiseja.
Merkinnöissä näitä on siitä syystä hyvä kuvata samoin keinoin.

Merkinnöissä moniulotteiset muuttujat erotellaan joskus yksiulotteisista joko lisäämällä muuttujan 
päälle nuoli tai käyttämällä paksunnettua fonttia. 
Syy erotteluun on enimmäkseen pedagoginen. Tärkeintä on, että käytetyt muuttujat on esitelty ennen niiden käyttöä. Lisäksi olisi huonon käytännön mukaista 
esimerkiksi nimetä jokin yksiulotteinen ja moniulotteinen muuttuja samalla merkillä niin, että erottavana tekijänä on vain päälle piirretty viiva.

Täsmällisin tapa esitellä tällainen muuttuja on kertoa millaiseen joukkoon se kuuluu. Esimerkiksi tavallinen yksiulotteinen realiluku ilmoitetaan kuuluvaksi 
realilukujoukkoon: \verb$x in \R$ ($x\in \mathbb{R}$). Vastaavasti $n$-ulotteiset realivektorit kuuluvat joukkoon \verb$\R^n$ ($\mathbb{R}^n$).  
Matriisien joukko puolestaan ilmaistaan dimensioiden ristitulona. 
Esimerkiksi $m$-rivinen ja $n$-sarakkeinen matriisi kuuluu joukkoon \verb$\R^(m \x n)$ ($\mathbb{R}^{m \times n}$).
Vaaka- ja pystysuuntaiset vektorit on mahdollista erotella sanallisesti.
Ne voi myös ilmaista matriiseina. 
Esimerkiksi kolmiulotteinen vaakavektori kuuluu joukkoon \verb$\R^(1 \x 3)$ ($\mathbb{R}^{1\times 3}$), ja kaksiulotteinen pystyvektori \verb$\R^(2 \x 1)$ 
($\mathbb{R}^{2\times 1}$).

Vektorit ja matriisit voidaan myös kirjoittaa auki niin, että alkiot ovat näkyvillä. Tähän tarkoitukseen käytetään hakasulkuja. Alkiot erotellaan toisistaan 
välilyönnein tai pilkuin. Rivin päättyminen ilmaistaan puolipistein. Joksus on myös kuuvavampaa tai helpompaa jakaa matriisi usealle riville, jolloin allekkaiset 
alkiot saadaan näkyviin allekkain.

\section{Derivointi}
Derivaattaa merkitään isolla D-kirjaimella derivaatan edessä. 
Jos halutaan määritellä, minkä muuttujan suhteen derivaatta on, lisätään tämä muuttuja heti D:n jälkeen.
Esimerkiksi x:n suhteen derivoitu \verb$x+y$ merkitään \verb$Dx x+y$.
Oletuksena on, että derivaattaan lasketaan kuuluvaksi kaikki D:n jälkeen tuleva. Jos derivaatta esiintyy kaavassa, sen voi erotella muista sulkein.

\section{Indeksijoukolliset operaattorit} \label{sec:indexsets}
Osassa operaatiosta tarkasteltavaksi otetaan jokin joukko muuttujan arvoja.
Mustavalkomerkinnöissä indeksijoukko tai väli ilmaistaan usein ylä- ja alaindeksien avulla.
Tämä olisi kelvolinen merkintätapa jo sellaisenaan, mutta tekisi usein merkinnöistä turhan monimutkaisia kirjoittaa ja lukea.
Siksi tässä luvussa esitetään vaihtoehtoisia tapoja.

\begin{itemize}
	\item Summa $\sum_{n=1}^\infty 2^(-n) = 1$: \verb$~S{n=1->\infty} 2^(-n) = 1$
	\item Tulo $n$:nnestä ensimmäisestä kokonaisluvusta eli $n$:n kertoma $\Pi_{i=1}^n i = n!$: \verb$\Pi{i=1->n} i = 1*2*...*n = n!$
	\item Integraali funktiosta $f$ väliltä $[a,b]$ $\int_a^b f(x) dx$: \verb$int{a,b} f(x) dx$
	\item Integraali funktiosta $f$ joukon $V$ yli $\int_V f dx$: \verb$int_V f dx$
	\item Yhdiste kaikista luonnollisen luvun sisältävistä yksiöistä on luonnollisten lukujen joukko $\bigcup_{i=1}^\infty \{i\} = \mathbb{N}$: 
	\verb$\U{i=1->\8} {i} = \N$
	\item Kaikkien välien $[0,x]$ leikkaus kaikilla positiivisilla realiluvuilla $x$ on $\bigcap_{x\in \mathbb{R}_+} [0,x] = \emptyset$:
	\verb$\I{x in \R_+} [0,x] = {}$
\end{itemize}

\section{Raja-arvot}
Raja-arvojen merkintä on hyvin samanlainen mustavalkomerkinnän ja latex-syntaksin kanssa.
Esimerkiksi lauseen $\frac{1}{x}$ raja-arvo, kun $x$ lähestyy ääretöntä eli $\lim(x \rightarrow \infty) \frac{1}{x} = 0$
merkitään \verb$lim(x->\infty) 1/x = 0$.
Toispuoleisissa raja-arvoissa kohdearvon jälkeen merkitään lähestymissuunta etumerkillä,
jolloin funktion $f$ oikean puoleinen (+-etumerkki) raja-arvo pisteessä \verb$x_0$ ($x_0$) on \verb$lim(x->x_0+) f(x)$.

\section{Kombinaatiot, binomikerroin}
Kombinatorisessa matematiikassa kombinaatio on joukon osajoukko ja $k$-kombinaatio puolestaan tasan $k$ alkiota sisältävä osajoukko.
Binomikerroin ${n \choose k}=\frac{n!}{k!(n-k)!}$ (lausutaan: "n yli k:n") ilmaisee $k$-kombinaatioiden määrän $n$-alkioisessa joukossa.
Tätä merkitään sulkeilla ja välilyönnillä: \verb$(n k)=n!/(k!(n-k)!)$.
Jos merkintä on monimutkaisempi, tilannetta voi selkiyttää lisäsulkein kuten yleensäkin (esimerkiksi: \verb$(n (k-1))$).
